\chapter{CONSIDERAÇÕES FINAIS}

Devido ao grande número de implementações dos sistemas de recomendação nas mais diversas áreas que aqui foram apresentadas, torna-se notável a vasta gama de aplicações das técnicas e, mais do que isso, a necessidade de um serviço multipropósito desprendido do uso de linguagens de programação específicas.

A aplicação desenvolvida supre as deficiências apontadas no capítulo \ref{intro}, alinhando as soluções desenvolvidas com os objetivos propostos na seção \nameref{objectives}. Desta forma, esta aplicação pode servir não só como uma alternativa \textit{open-source} aos sistemas de recomendação, mas também como uma tecnologia de utilização simples para futuros estudos na área.

Além disso, é importante ressaltar a preocupação ao longo de todo o processo com o desempenho geral da API, a qualidade da documentação gerada e, principalmente, a qualidade das recomendações. Visando ampla utilização, esta API tem um poder de personalização superior ao visto nos outros trabalhos apresentados na seção \nameref{related_work}, sendo este um diferencial para os usuários. Este diferencial também proporciona algumas vantagens, tais como a escolha de quais atributos serão utilizados pelo sistema recomendados e, posteriormente, qual estrutura será retornada ao usuário.

Não menos importante do que a qualidade das recomendações, está a orientação a metadados, ponto chave durante o desenvolvimento e utilização da aplicação. A escolha de tornar toda a parte central da aplicação personalizável é de grande valhia para listagem de inúmeras possibilidades diferentes de uso e, consequentemente, inúmeros utilizadores em potencial.

Por fim, a aplicação possui seus pontos de melhora, a fim de torná-la cada vez mais precisa nas recomendações e simples ao usuário final.
