\chapter{RESULTADOS E DISCUSSÃO}

Após a implementação de todas as funcionalidades descritas ao longo deste trabalho, todos os objetivos inicialmente traçados foram atingidos. A aplicação gera, de forma híbrida, recomendações a usuários e itens, levando em consideração quaisquer atributos e pesos definidos pelo usuário, além da possibilidade de personalização completa da estrutura de envio e retorno dos dados.

Também é possível acessar a documentação da aplicação, contendo todos os módulos de forma hierárquica e organizada, visando facilitar o compartilhamento futuro com a comunidade. Esta documentação, anexa a todo o código fonte da aplicação, está acessível através do repositório da aplicação no Github: \url{github.com/iurykrieger96/morpy}.

O protótipo da API resultante deste trabalho foi denominada \textbf{MORPY} - \textit{\textbf{M}etadata \textbf{O}riented \textbf{R}ecommendations for \textbf{PY}ton}, visando resumir todas as principais funcionalidades e diferenciais da mesma em um acrônimo.

No entanto, mesmo com os objetivos principais atingidos, trabalhos pendentes precisam ser efetuados para assegurar melhoras de performance e usabilidade da aplicação. Como trabalhos futuros a serem efetuados podem ser listados os seguintes:

\begin{itemize}
	\item O término do processo de documentação nos módulos faltantes;

	\item A utilização de \textit{machine learning} nao só na recomendação de itens mas também na configuração dos metadados, encontrando o melhor conjunto de itens para serem levados em consideração pelo motor de recomendações;

	\item A otimização do processo executado pelo motor de recomendações, fazendo com que processe as matrizes de similaridade e distância utilizando a \textbf{GPU}.

	\item A implementação de autenticação via \textit{token} utilizando um padrão mais seguro que o atual.

	\item O desenvolvimento de uma página de apresentação da aplicação, agregando uma descrição das funcionalidades da API, sua documentação e um link para o Github.
\end{itemize}