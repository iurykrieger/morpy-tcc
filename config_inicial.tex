% ---
% Pacotes fundamentais 
% ---
\usepackage{cmap}				% Mapear caracteres especiais no PDF
\usepackage{times}			    % Usa a fonte Times
\usepackage[T1]{fontenc}		% Selecao de codigos de fonte.fontenc
\usepackage[utf8]{inputenc}		% Codificacao do documento (conversão automática dos acentos)
\usepackage{lastpage}			% Usado pela Ficha catalográfica
\usepackage{indentfirst}		% Indenta o primeiro parágrafo de cada seção.
\usepackage{color}				% Controle das cores
\usepackage{graphicx}			% Inclusão de gráficos
\usepackage{amsfonts}			% Símbolos

%----Ajuste no alinhamento das listas
\usepackage{enumitem}
\setitemize[0]{itemindent=0.4cm,itemsep=0pt}
\setenumerate[0]{itemindent=0.5cm,itemsep=0pt}

%------
% ---
% Pacotes de citações
% ---
\usepackage[brazilian,hyperpageref]{backref}	 					  % Paginas com as citações na bibl
\usepackage[a4paper]{geometry}
%Referência
\usepackage[
	alf, 	
	abnt-emphasize=bf,
	abnt-url-package=none,
	abnt-repeated-title-omit=yes,
	abnt-full-initials=yes,                                        %yes nome por extenso, no apenas iniciais
	abnt-etal-list=3												%abreviar com mais de 3 autores
]{abntex2/abntex2cite}				 														    % Citações padrão ABNT
\usepackage{lipsum}							   								       % para geração de dummy text
\usepackage{multirow}
\usepackage{float}
\usepackage{setspace}
\usepackage{tabularx}

%\usepackage[none]{hyphenat}
%\captionsetup[table]{justification=raggedright}
% Configurações de aparência do PDF final
% alterando o aspecto da cor azul

\usepackage{listings}
\usepackage{xcolor}
\usepackage{caption}
\definecolor{blue}{RGB}{41,5,195}
\definecolor{codegreen}{rgb}{0,0.6,0}
\definecolor{codegray}{rgb}{0.5,0.5,0.5}
\definecolor{codepurple}{rgb}{0.58,0,0.82}
\definecolor{backcolour}{rgb}{0.95,0.95,0.92}

\lstdefinestyle{custompython}{
	backgroundcolor=\color{backcolour},
    commentstyle=\color{codegreen},
    keywordstyle=\color{codegreen},
    numberstyle=\tiny\color{codegray},
    stringstyle=\color{codepurple},
    basicstyle=\footnotesize,
    breakatwhitespace=false,
    breaklines=true,
    captionpos=b,
    keepspaces=true,
    numbers=left,
    numbersep=5pt,
    showspaces=false,
    showstringspaces=false,
    showtabs=false,
    tabsize=2
}

% --- 
% Espaçamentos entre linhas e parágrafos 
% --- 
% O tamanho do parágrafo é dado por:
\setlength{\parindent}{2cm}
\linespread{1.5}

%Espaçamento depois dos títulos
\setlength{\afterchapskip}{\baselineskip}
% %\setlength{\afterchapskip}{\lineskip}

% Controle do espaçamento entre um parágrafo e outro:
\setlength{\parskip}{0cm}  % tente também \onelineskip

\hangcaption
\captionstyle[\raggedright]{}

%Estava mostrando nas referencias quais paginas estavam sendo referenciadas
\renewcommand{\backref}{}
\renewcommand*{\backrefalt}[4]{}

%Reduzir a fonte do caption
%\captionnamefont{\centering\ABNTEXfontereduzida}
%\captiontitlefont{\centering\ABNTEXfontereduzida}
%Ajuste nas listas de tabela, ilustrações e quadros
\setlength\cftbeforechapterskip{0pt}
% ---
% compila o indice
% ---
\makeindex
% ---