\documentclass[12pt, openright, oneside, a4paper, brazil]{abntex2}
\usepackage{lmodern}
\usepackage[T1]{fontenc}		% Selecao de codigos de fonte.
\usepackage[utf8]{inputenc}
\usepackage{nameref}
\usepackage[justification=justified,singlelinecheck=false]{caption}
\usepackage{lastpage}
\usepackage{amsmath}
\usepackage{indentfirst}
\usepackage{color}				% Controle das c ores
\usepackage{graphicx}			% Inclusão de gráficos
\usepackage{microtype} 			% para melhorias de justificação
\usepackage[brazilian,hyperpageref]{backref}	 % Paginas com as citações na bibl
\usepackage[alf, abnt-emphasize=bf, abnt-url-package=none, abnt-repeated-title-omit=yes, abnt-full-initials=yes, abnt-etal-list=3, abnt-etal-text=emph]{abntex2cite} % Citações padrão ABNT
\usepackage{trabalho}	% Citações padrão ABNT
\usepackage{lipsum}	% para geração de dummy text
\usepackage{trivfloat}	%Quadros
% ---

% ---
% Pacotes de citações
% ---

% ---
% CONFIGURAÇÕES DE PACOTES

% Quadro
\trivfloat{quadro}
\renewcommand{\listquadroname}{Lista de Quadros}
\renewcommand{\backrefpagesname}{Citado na(s) página(s):~}
\renewcommand{\backref}{}
\renewcommand*{\backrefalt}[4]{
	\ifcase #1 %
		Nenhuma citação no texto.%
	\or
		Citado na página #2.%
	\else
		Citado #1 vezes nas páginas #2.%
	\fi}%

\titulo{UMA API WEB ORIENTADA A METADADOS COMO SERVIÇO DE RECOMENDAÇÃO HÍBRIDA}
\autor{Iury Krieger}
\local{\vfill Videira - Santa Catarina}
\data{2017}
\orientador{Msc. Tiago Heineck}
\coorientador{Msc. Wanderson Rigo}
\instituicao{
  Instituto Federal Catarinense - Câmpus Videira
  \par
  Bacharelado em Ciência da Computação
}
\tipotrabalho{Trabalho de conclusão de curso}
\preambulo{
	Trabalho de conclusão de curso submetido ao Instituto Federal Catarinense - Câmpus Videira como parte dos requisitos para a obtenção do grau de Bacharel em Ciência da Computação
}
\makeatletter
\hypersetup{
	pdftitle={\@title},
	pdfauthor={\@author},
	pdfsubject={\imprimirpreambulo},
	pdfcreator={LaTeX with abnTeX2},
	pdfkeywords={abnt}{latex}{abntex}{abntex2}{trabalho acadêmico},
	colorlinks=true,       		% false: boxed links; true: colored links
	linkcolor=blue,          	% color of internal links
	citecolor=blue,        		% color of links to bibliography
	filecolor=magenta,      		% color of file links
	urlcolor=blue,
	bookmarksdepth=4
}
\makeatother
\setlength{\parindent}{1.3cm}
\setlength{\parskip}{0.2cm}
\makeindex

\begin{document}
\frenchspacing
\imprimircapa
\imprimirfolhaderosto

\begin{comment}
\begin{folhadeaprovacao}
	\begin{figure}[htb]
		\begin{center}
			\includegraphics{images/logo.png}
		\end{center}
	\end{figure}
	{\ABNTEXchapterfont\large{\textbf{BACHARELADO EM CIÊNCIA DA COMPUTAÇÃO}}}
	\begin{center}
		{\ABNTEXchapterfont\large\imprimirautor}

		\vspace*{\fill}\vspace*{\fill}
		\begin{center}
			\ABNTEXchapterfont\bfseries\Large\imprimirtitulo
		\end{center}
		\vspace*{\fill}
	\end{center}

	Este Trabalho de Conclusão de Curso foi julgado adequado para a obtenção do título de Nome do Título em Nome do Curso, Área de Concentração e aprovada em sua forma final pelo Curso de Nome do Curso

	\assinatura{\textbf{\imprimirorientador} \\ Orientador}
	\assinatura{\textbf{Msc. Marcelo Cendron} \\ Professor Convidado I}
	\assinatura{\textbf{Maurício Ferreira} \\ Professor Convidado II}

	\begin{center}
		\vspace*{0.5cm}
		{\large\imprimirlocal}
		\par
		{\large\imprimirdata}
		\vspace*{1cm}
	\end{center}
\end{folhadeaprovacao}
\end{comment}

\begin{folhadeaprovacao}

  \begin{center}
  \vspace*{-1.2cm}
    {\large\imprimirautor}

    \vspace*{\fill}\vspace*{\fill}\vspace*{\fill}
    {\large\imprimirtitulo}
    \vspace*{\fill}\vspace*{\fill}

    \hspace{.45\textwidth}
    \begin{minipage}{.5\textwidth}
        \imprimirpreambulo
    \end{minipage}%
    \vspace*{\fill}
   \end{center}

  \begin{center}
  	 Videira (SC), 16 de Maio de 2017
  \end{center}

    \vspace{-1cm}

   \assinatura{\begin{center}\vspace{-0.6cm}\imprimirorientador \\
   					   Instituto Federal Catarinense
   					   \end{center}
   	}
   	\assinatura{\begin{center}\vspace{-0.6cm} Msc. Wanderson Rigo \\
   					   Instituto Federal Catarinense
   					   \end{center}
   	}
    \begin{center}
  	\textbf{BANCA EXAMINADORA}
   \end{center}
   \vspace{-1cm}
   \assinatura{\begin{center}\vspace{-0.6cm} Msc. Marcelo Cendron \\
      					   Instituto Federal Catarinense
   					     \end{center}
   }
   \assinatura{\begin{center}\vspace{-0.6cm}Maurício Ferreira \\
       					   Instituto Federal Catarinense
   					    \end{center}
    }

    \vspace*{1cm}

\end{folhadeaprovacao}

\setlength{\absparsep}{18pt} % ajusta o espaçamento dos parágrafos do resumo
\begin{resumo} 

Devido a expansão massiva de dados produzidos e disponíveis na Internet, os usuários estão cada vez mais sobrecarregados de informação, não sabendo distinguir informações realmente úteis. Para sanar este problema, os sistemas de recomendação visam recomendar os itens mais úteis a cada usuário, através de técnicas de machine learning. Tais técnicas visam prever a avaliação de um usuário a um item, baseando-se nas avaliações já conhecidas. Este trabalho propõe o desenvolvimento de uma API Web de código aberto que recomenda itens a usuários, fazendo uso de um sistema de recomendação híbrido que analisa as estruturas pré definidas e proporciona recomendações, baseando-se nos metadados fornecidos, através do conteúdo do item e da filtragem colaborativa de usuários. Tal sistema poderá processar suas recomendações utilizando a GPU, minimizando o tempo da requisição de recomendação e consequentemente aumentando a eficiência da aplicação. Dessa forma é possível fornecer um serviço multi-propósito desprendido de qualquer ambiente e linguagem de programação, trazendo uma visão mais transparente dos sistemas de recomendação aos desenvolvedores.

\textbf{Palavras-chaves}: Sistemas de Recomendação. Aprendizado de Máquina. Metadados. Computação em GPU.
\end{resumo}

\begin{resumo}[Abstract]
 \begin{otherlanguage*}{english}

Due to the massive expansion of data produced and available on the Internet, users are increasingly overloaded with information, not knowing how to distinguish which is really useful. To remedy this problem, recommendation systems aim to recommend the most useful items to each user through machine learning techniques. These techniques are intended to predict a user's rating of an item, based on previously known rating. This work proposes the development of an open-source Web API that recommends items to users, making use of a hybrid recommendation system that analyzes the pre-defined structures and provides recommendations, based on the metadata provided, through item content and collaborative filtering. Such a system can process its recommendations using the GPU, minimizing the time of the recommendation request and consequently increasing the application efficiency. Therefore is possible to provide a multi-purpose service detached from any environment and programming language, bringing a more transparent view of recommendation systems to developers.

  \textbf{key-words}: Recommender Systems. Machine Learning. Metadata. GPU Computing.
 \end{otherlanguage*}
\end{resumo}

% ---
% Lista de quadros
% ---
\counterwithout{quadro}{chapter}
\newpage % Forçar a lista de quadros em uma nova página
\phantomsection % Comando necessário caso o pacote hyperref seja utilizado, visando a corrigir o link
\pdfbookmark[0]{\listofquadros}{lof}
\listofquadros* % Adiciona lista de quadros
\cleardoublepage

% ---
% Lista de figuras
% ---
\pdfbookmark[0]{\listfigurename}{lof}
\listoffigures*
\cleardoublepage

% ---
% Siglas
% ---
\begin{siglas}

	\item[IA]{\textit{Inteligência Artificial}}
	\item[HTTP]{\textit{Hypertext Transfer Protocol}}
	\item[API]{\textit{Application Program Interface}}
	\item[REST]{\textit{Representational State Transfer}}
	\item[JSON]{\textit{Javasript Object Notation}}

\end{siglas}

% ---
% Sumário
% ---
\pdfbookmark[0]{\contentsname}{toc}
\tableofcontents*
\textual

\chapter{INTRODUÇÃO}

Com o avanço crescente do campo tecnológico, os computadores vêm desempenhando tarefas antes incumbidas à seres humanos. O poder de computação provou-se muito eficaz ao desempenhar tarefas que possuíssem um padrão possível de se expressar através de um algoritmo, mais ainda, se este padrão fosse repetitivo.

Logo os computadores começaram a desempenhar funções nas mais diversas áreas, desde cálculos matemáticos à manipulação de imagens. Atualmente, das funções desempenhadas pelos computadores, a mais difícil de se reproduzir com precisão é o padrão de raciocínio humano.

Alguns autores defendem que para que um computador atinja tal nível, seria necessário que o mesmo possuísse consciência, assim como os seres humanos. Outros defendem que o raciocínio humano não consegue ser reproduzido, apenas emulado, devido à impossibilidade de se programar uma consciência computacional. Tal área de estudo, que tem como o foco o desenvolvimento de sistemas computacionais rumo a proximidade do método humano, chama-se inteligência artificial \cite{russell2004inteligencia, coppin2015inteligencia}.

Possível ou não, é inegável o avanço da inteligência artificial desde seu início nos primórdios da computação. Algumas tarefas, tais como a atribuição de uma consciência a um sistema computacional, deixaram de ser o foco da área, uma vez que não possuímos a tecnologia para construir sistemas muito mais complexos que os atuais  \cite{russell2004inteligencia}.

Entretanto, a inteligência artificial encontrou-se muito eficaz em outras áreas do método humano, tais como o aprendizado, um dos segmentos mais importantes da área, dentro da inteligência artificial chamado de aprendizado de máquina (\textit{machine learning}) \cite{coppin2015inteligencia}.

Desde os anos 90 a preocupação com o armazenamento e a expansão massiva de dados produzidos já existia, prevendo que usuários ficariam sobrecarregados de informação, não sabendo distinguir o que seria realmente útil \cite{hill1995recommending, adomavicius2005toward}. Na época, uma comunidade virtual de avaliação foi proposta para proporcionar aos usuários o mínimo de esforço ao encontrar informações úteis. Com a evolução da inteligência artificial e das técnicas de machine learning, este trabalho de avaliação e recomendação, antes feito por uma comunidade, hoje é atribuído aos sistemas de recomendação \cite{hill1995recommending}.

Sistemas de recomendação (RSs) são ferramentas de  software e técnicas que provém sugestões de artefatos à usuários. Estes artefatos são definidos como os objetos de valor à serem recomendados \cite{ricci2011introduction}. Atualmente, o interesse em tais sistemas se mantém alto, devido a abundância de aplicações práticas \cite{adomavicius2005toward}, exemplificadas nos casos de \textit{E-commerce} por \citeonline{schafer2001commerce}, além de \citeonline{linden2003amazon}, onde são amplamente utilizados.

Desta forma, sistemas de recomendação vem sendo desenvolvidos para a resolução do problema descrito nas mais diversas áreas \cite{bennett2007netflix, gavalas2014mobile}, desde aplicações hoteleiras como o TripAdvisor até aplicações de entretenimento como a Netflix, além da sua origem nos \textit{E-commerces} citados anteriormente. Muitos destes sistemas são casos de RSs aplicados a itens e finalidades específicas \cite{huang2002graph, brozovsky2007recommender}, onde todo o motor de recomendação segue uma abordagem baseada no padrão que lhe foi dado.

Por outro lado, ao observar aplicações web de sistemas de recomendação, verifica-se a existência de soluções em forma de APIs, tais como o Google Cloud Platform e o Microsoft Cognitive Services, fornecidas como serviços transparentes. Entretanto, estas soluções proprietárias não são incorporadas a aplicação, mas sim utilizadas como serviços externos, dificultando a personalização.

Para sanar estes problemas, este trabalho propõe o desenvolvimento de uma API que proporcione uma visão mais transparente dos sistemas de recomendação, permitindo ao usuário desfrutar das funcionalidades, sem a necessidade de um profundo conhecimento dos detalhes que compõem as diferentes técnicas de recomendação, além dos problemas decorrentes do uso de cada uma das técnicas. Além disso, tal tecnologia será fornecida como um serviço de código aberto, podendo ser utilizada em qualquer ambiente.

Este trabalho está dividido em seis seções. A segunda seção apresenta o referencial teórico necessário para o entendimento total do escopo do trabalho. Na terceira seção são apresentadas as principais características do trabalho proposto, além de compará-lo com outros trabalhos relacionados. Em seguida, a quarta seção apresenta a metodologia a ser utilizada para realização do trabalho proposto na seção anterior. Mais à frente, na seção cinco, será abordado o cronograma a ser empregado para a realização do trabalho e, por fim, na sexta seção são apresentadas as considerações finais.

\section{Objetivos}

\subsection{Objetivo Geral}

Desenvolver uma API web de código aberto para recomendação híbrida de itens a usuários.

\subsection{Objetivos Específicos}

\begin{itemize}

	\item Fornecer uma documentação das funcionalidades visando futura colaboração da comunidade e utilização por outros desenvolvedores.
	\item Proporcionar a recomendação das propriedades relevantes através das estruturas de metadados fornecidas.

\end{itemize}

\section{Metodologia}

A metodologia deste trabalho está dividida em três seções. Primeiramente serão implementadas todas as funcionalidades descritas anteriormente. Mais à frente, será feita a validação das funcionalidades implementadas e da eficácia das recomendações. Por fim, serão feitos os ajustes e correções necessárias de acordo com o resultado da validação das funcionalidades implementadas.

\subsection{Implementação}

Inicialmente serão implementados os algoritmos de recomendação híbrida, incluindo o processamento dos metadados fornecidos. Os algoritmos de recomendação resumem a eficácia da API e devem consumir a maior parte do tempo de desenvolvimento.

Ao completar a implementação das técnicas híbridas de sistemas de recomendação, serão implementadas as demais funcionalidades da API. Serão consideradas a identificação e entrada dos metadados, além do formato dos dados de saída.

Ao fim do desenvolvimento, o código será adaptado para processamento na GPU através de diretivas de compilação e bibliotecas especializadas.

\subsection{Validação}

Assim que a API esteja em um grau considerado funcional, será feita a validação da eficácia ao recomendar as estruturas fornecidas através de grupos de testes definidos, uma técnica amplamente utilizada na validação de técnicas de machine learning.

A validação será feita utilizando um grupo separado dos  dados utilizados para testes, confrontando as recomendações feitas com o resultado esperado. Através desses resultados é medida a acurácia de um sistema de recomendação, métrica utilizada como medida de eficiência entre os diferentes métodos utilizados.

\subsection{Ajustes e Correções}

Por fim, serão feitos os ajustes e correções de erros recolhidos ao longo do processo, além de testar as funcionalidades e a utilização da API como um todo. A documentação será feita durante boa parte de todo o processo e, neste caso em específico, possui um foco especial, uma vez que o princípio da API é que a mesma seja utilizável por outros desenvolvedores, além de possibilitar a contribuição da comunidade.

\section{Trabalhos Relacionados}

Tendo como base as técnicas descritas acima, existem trabalhos como os apresentados por \citeonline{guo2015librec}, que abordam as técnicas em forma de biblioteca Java a ser incluída nos projetos. Esta abordagem torna a utilização mais simples devido ao fato do usuário poder utilizar apenas as funcionalidades da biblioteca, preocupando-se com o formato de entrada e saída dos dados, não com o processo de recomendação em si. Outra abordagem interessante é a proposta por \citeonline{brozovsky2007recommender} ao construir uma biblioteca C\# multi-propósito, focando na recomendação de itens com base na avaliação em um esquema de \textit{rating} (de uma a cinco estrelas), ou com base apenas em itens com avaliação positiva.

Do mesmo modo que a biblioteca desenvolvida por \citeonline{brozovsky2007recommender}, a API proposta neste trabalho também visa ser multi-propósito e distribuída como código aberto pela licença pública GNU (\textbf{GPL}), porém, fornecendo tais funcionalidades como um serviço web independente de linguagem de programação, o que não acontece nos exemplos apresentados.

Além dos trabalhos apresentados, \citeonline{do2013filtragem} aborda os sistemas de recomendação com uma perspectiva semelhante a este trabalho, focando mais no ganho de desempenho ao processar o método de filtragem colaborativa na GPU. Este trabalho não tem seu foco em desempenho, mas sim em uma proposta de \textbf{recomendação genérica}, que forneça recomendações a quaisquer modelos de usuários e itens através do método híbrido.

%
%--------- FIM INTRODUÇÃO------------
%

\cleardoublepage

\chapter{REFERENCIAL TEÓRICO}

\section{Aprendizado de Máquina}

Um dos segmentos da inteligência artificial com grande importância na atualidade é o aprendizado de máquina. Responsável pela construção de agentes capazes de, a partir de uma coleção de pares de entrada e saída, aprender uma função que prevê a saída para novas entradas. Tais agentes são definidos como tudo que pode perceber seu ambiente através de sensores, além de atuar sobre o mesmo através de atuadores. Em outras palavras, o aprendizado de máquina resume-se em técnicas que proporcionam a um algoritmo a capacidade de melhorar seu desempenho de forma automática, através do conhecimento obtido pelas entradas existentes \cite{coppin2015inteligencia}.

Dessa forma, considera-se que um agente está aprendendo se melhorar o seu desempenho nas tarefas para que foi designado, a partir de suas observações sobre o mundo. Este aprendizado proporciona às técnicas de \textit{machine learning} a capacidade evolutiva, uma vez que é possível não só responder as entradas do mundo exterior como também tirar conclusões sobre as mesmas, melhorando cada vez mais a natureza da solução \cite{russell2004inteligencia}.

Conforme apresentado por \citeonline{carbonell1983overview}, devido a capacidade de, além de solucionar problemas, melhorar automaticamente o desempenho da solução, os sistemas de aprendizagem tem suas aplicações nas mais diversas áreas, tais como agricultura, educação, sistemas especialistas de alta performance, reconhecimento de imagem, programação, etc. Através de um apanhado das aplicações nas áreas de utilização, \citeonline{carbonell1983overview} dividem o campo de aprendizado do \textit{machine learning} em três partes:

\begin{itemize}

	\item \textbf{Estudos orientados à tarefa (\textit{Task-oriented studies}):} composto pelo desenvolvimento e análise de sistemas de aprendizagem visando melhorar a performance na solução de determinadas tarefas.

	\item \textbf{Simulação cognitiva (\textit{Cognitive simulation}):} formado pela investigação e simulação do processo de aprendizagem humano.

	\item \textbf{Análise teórica (\textit{Theoretical analysis}):} exploração teórica do espaço de possíveis processos de aprendizado.

\end{itemize}

Analisando a taxonomia proposta por \citeonline{carbonell1983overview}, pode-se idenficar que o escopo deste trabalho encontra-se nos estudos orientados à tarefa, onde o propósito é a melhoria da performance, neste caso através de recomendações orientadas à metadados.

Como exemplo do uso das técnicas de \textit{machine learning}, \citeonline{sebastiani2002machine}  apresenta um algoritmo de categorização de texto que, a partir de um conjunto de documentos pré-classificados (entradas), constrói um classificador para novos documentos (novas entradas). Outro exemplo, apresentado por \citeonline{pang2002thumbs}, reforça a ideia de melhora de desempenho para novas entradas através de um padrão aprendido a partir de entradas já existentes. Através de dados sobre avaliações de filmes, pode-se perceber que, mesmo as técnicas padrão de \textit{machine learning}, acabam superando os patamares humanos na classificação de sentimentos.

\section{Sistemas de Recomendação} \label{recommender_systems}

Como ramificação do aprendizado de máquina, os sistemas de recomendação (RSs) são técnicas de software que provém sugestões a usuários de itens que os mesmos possam querer utilizar   \cite{resnick1997recommender, schafer1999recommender}. Desta forma, recomendações seriam, em sua forma mais simples, rankings de itens, tais como os utilizados na maioria das soluções de produtos (livros mais lidos, filmes mais assistidos, etc.) \cite{ricci2011introduction}. O que os RSs trazem de novo é a tentativa de predizer, através da filtragem colaborativa ou da similaridade de conteúdo, qual o ranking mais adequado de produtos ou serviços a um usuário. A filtragem colaborativa, termo cunhado por \citeonline{resnick1997recommender}, recomenda itens baseando-se nos relacionamentos do usuário. Por outro lado, a similaridade de conteúdo baseia-se no conteúdo de itens já avaliados pelo usuário. Tais dados podem ser coletadas de forma explícita, na forma de perguntas diretas e avaliações do usuário sobre os itens, ou de forma interpretativa, inferindo sobre ações tomadas pelo usuário e atribuindo peso a elas.

Mais formalmente, os sistemas de recomendação podem ser descritos matematicamente da seguinte forma: sendo \emph{C} o conjunto de todos os usuários e \emph{S} o conjunto de todos os itens que podem ser recomendados, tanto o espaço \emph{S} como o espaço \emph{C} podem ser extremamente grandes, batendo os milhões de usuários e itens \cite{adomavicius2005toward, gomez2016netflix}. Dessa forma, tem-se \textit{u} como a função de utilidade de um item \emph{s} para um usuário c. A função u utiliza-se do conjunto ordenado R, descrito como $C \times S \rightarrow R$, para encontrar o item $\emph{s} \in \emph{S}$ com a maior utilidade para o usuário \emph{c}. Um exemplo de como as preferências são armazenadas no espaço de avaliações C x S pode ser visto no quadro 1.

\begin{quadro}[h!tp]

	\caption{\label{movie_matrix}Exemplo de matriz de recomendações a filmes}

	\begin{center}
		\includegraphics[scale=0.8]{images/movie_matrix.png}
	\end{center}

	\hspace{5.5cm}{Fonte: \citeonline{adomavicius2005toward}}

\end{quadro}

De acordo com o quadro 1, o símbolo "$\emptyset$" representa os filmes ainda não avaliados pelos usuários. Estes itens, por sua vez, são os alvos das técnicas de recomendação que tentam predizer a avaliação de um usuário. Uma vez que o motor de recomendação pode predizer as avaliações de um usuário, pode-se recomendar ao mesmo apenas os \emph{N} itens com a maior avaliação estimada \cite{adomavicius2005toward}.

Como consequência da importante participação dos sistemas de recomendação em sites com um grande número de público, tais como Netflix, eBay e Amazon.com, os mesmos tornaram-se ferramentas poderosas \cite{schafer1999recommender} e são considerados os propulsores de várias estatísticas, entre elas: o aumento da satisfação dos usuários, devido a precisão das recomendações; o aumento da fidelidade dos usuários, devido ao aumento de precisão quanto maior for a interação do usuário com o site; a aumento da capacidade do próprio serviço em entender melhor as intenções de seu público \cite{ricci2011introduction}. Tendo em vista o crescimento do número de aplicações que utilizam sistemas de recomendação e da variedade de soluções utilizadas em grandes sites, torna-se notável a importância dos mesmos.

Como próximo passo na evolução dos sistemas de recomendação, \citeonline{adomavicius2015context} propõem que os RSs, além de considerarem a similaridade entre perfis, devem estar cientes do contexto da avaliação do usuário ao construírem o modelo de perfil. Chamados de sistemas cientes de contexto, estes sistemas de recomendação devem diferenciar a ação que o usuário toma ao apenas analisar um item (filme, produto, etc.), não necessariamente indicando que itens parecidos devem ser recomendados no futuro, da ação tomada ao consumir um item (comprar, assistir, etc.). A partir dessa distinção de contexto, os RSs poderiam atribuir pesos diferentes para cada ação, podendo assim fazer recomendações mais precisas.

A seguir serão apresentados as diferentes técnicas dos sistemas de recomendação, além de qual técnica será utilizada por este trabalho e seus diferentes métodos através de algoritmos. Devido a existência de inúmeras técnicas e métodos de recomendação, este trabalho abordará apenas as técnicas necessárias para entendimento do mesmo, aprofundando-se apenas nos métodos que compõem a técnica utilizada.

\subsection{Método Baseado em Conteúdo}

Sistemas de recomendação que implementam o método baseado em conteúdo (\textit{content-based}) analisam um conjunto de documentos/descrições de itens previamente avaliados pelo usuário, construindo um modelo dos interesses baseando-se nas características dos itens avaliados \cite{mladenic1999text, adomavicius2005toward, lops2011content}. Este modelo serve para ser cruzado com o conteúdo de outros itens ainda não avaliados pelo usuário. Quanto maior o grau de semelhança entre o modelo do usuário e as características do item, maior a probabilidade do mesmo ter interesse.

Para que o modelo de interesses do usuário seja criado e confrontado com outros conteúdos ainda não avaliados, são necessários três atores principais que dividem a recomendação baseada em conteúdo: \textbf{analisador de conteúdo}, \textbf{aprendiz de perfis} e \textbf{componente de filtragem} \cite{lops2011content}. A estrutura completa destes agentes pode ser vista na \autoref{content_based}.

\begin{figure}[h!tp]
	\caption{\label{content_based}Arquitetura de um sistema baseado em conteúdo}

	\begin{center}
		\includegraphics[scale=0.75]{images/content_based.png}
	\end{center}

	\hspace{5.5cm}{Fonte: \citeonline{lops2011content}}
\end{figure}

Note que na \autoref{content_based}, a primeira parte do processo começa com o \textbf{analisador de conteúdo} (\textit{content analyzer}), transformando dados não estruturados em estruturas de atributos e características \cite{lops2011content, mladenic1999text}, armazenando-as no repositório de itens representados (\textit{represented items}). Para a construção e atualização do perfil de interesses do usuário ativo (representado na \autoref{content_based} por $u_{a}$), as avaliações do usuário para novos itens são armazenadas no repositório de feedback. O tipo de avaliação depende de cada aplicação, podendo ser expressado de forma \textbf{explícita}, como as avaliações binárias (\textit{like/dislike}) e avaliações em forma de rating (0 a 5; 1 a 5 estrelas) utilizadas em muitos sites, ou mesmo por avaliações \textbf{implícitas}, onde uma ação sobre um item (seleção, por exemplo) possui um peso atribuído \cite{pazzani2007content}.

De posse do repositório de itens representados, o \textbf{aprendiz de perfis} varre os itens $I_{k}$ do usuário ua em prol de construir o conjunto treinamento $TR_{a}$. O conjunto de treinamento é um conjunto de pares $\lbrace I_{k}$, $r_{k} \rbrace$, onde $r_{k}$ é a avaliação dada pelo usuário ua a representação do item $I_{k}$. Após a construção do conjunto de treinamento $TR_{a}$, o \textbf{aprendiz de perfis} aplica algoritmos de aprendizagem supervisionada para gerar o modelo de interesses do usuário $u_{a}$. Os modelos de interesses são armazenados no repositório de perfis (representado na \autoref{content_based} por \textit{profiles}) para uso futuro pelo \textbf{componente de filtragem}.

Quando a representação de um novo item é adicionada ao conjunto de itens representados, o componente de filtragem prediz se o mesmo será de interesse do usuário $u_{a}$, atraveś da comparação entre os atributos e características do novo item e o modelo de interesses do usuário. Em consequência, o componente de filtragem ranqueia os itens com os maiores potenciais de interesse, agrupando-os em uma lista de recomendações $L_{a}$ e apresentando-a ao usuário $u_{a}$. Dessa forma o usuário ua pode prover novas avaliações (\textbf{feedbacks}) dos itens da lista $L_{a}$, fazendo com que o aprendiz de perfis atualize seu modelo de interesses através da reconstrução do conjunto de treinamento $TR_{a}$ \cite{lops2011content}.

Atualmente \citeonline{pazzani2007content} apresentam que, devido ao grande crescimento de informação disponível para treinamento, os métodos atuais reduzem o conjunto de treinamento para algumas centenas de linhas, porém altamente relevantes (através de técnicas como o TF-IDF \footnote{Frequência do termo inverso da frequência nos documentos. Medida estatística para indicar a importância de uma palavra de um documento em relação a uma coleção de documentos. É frequentemente usada na mineração de dados.}). Dessa forma, por mais que as bases de dados aumentem, o conjunto de treinamento se mantém relevante e não é necessário percorrer todo o conjunto ordenado $R$.

\subsection{Método Baseado em Colaboração}

O método de filtragem colaborativa (\textit{collaborative-based} - CF) baseia-se no processo de avaliar itens através da opinião de outras pessoas. Tal processo, que começou com a filtragem da natureza de repositórios de texto, passou a ser mais informal, abrangendo até listas de discussão e arquivos de \textit{e-mail}. No começo, usuários tinham que acessar sites específicos, tais como o MovieLens, para receberem recomendações de filmes. Conforme os sistemas baseados em CF foram se popularizando, os sites começaram a utilizar estes sistemas para adequar seu conteúdo para cada usuário \cite{schafer2007collaborative}.

Assim como os sistemas baseados em conteúdo, sistemas de filtragem colaborativa também levam em consideração as avaliações de itens (mesmo que de outros usuários similares), através dos métodos de avaliação já descritos. Segundo \citeonline{adomavicius2005toward}, a diferença entre estes dois processos existe pelo fato de que a utilidade $u(c,s)$ de um item $s$ a um usuário $c$ é medida não pela utilidade $u(c,s_{i})$ dos itens $s_{i}$ similares ao item $s$, mas sim pela utilidade $u(c_{j}, s)$ do item $s$ baseado nos usuários $c_{j}$ \textbf{similares} ao usuário $c$. Em outras palavras, na filtragem colaborativa, os itens considerados úteis a um usuário são os itens úteis a usuários similares a ele.

Partindo desta premissa, \citeonline{sarwar2001item} abordam os sistemas de filtragem colaborativa a partir do seguinte cenário: uma lista de $\textbf{m}$ usuarios $U \lbrace u_{1}, u_{2}, …, u_{m} \rbrace$ e uma lista de $\textbf{n}$ itens $I \lbrace i_{1}, i_{2}, …, i_{n} \rbrace$. Cada usuário $u_{i}$ possuindo uma lista $Iu_{i}$ de itens, avaliados ou não. Conforme na \autoref{collaborative_based}, o algoritmo de filtragem colaborativa (\textbf{CF}) opera sobre a matriz de avaliações $n \times m$.

\begin{figure}[htb]
	\caption{\label{collaborative_based}Processo de recomendação colaborativa.}

	\begin{center}
		\includegraphics[scale=0.6]{images/collaborative_based.png}
	\end{center}
	\hspace{5.5cm}{Fonte: \citeonline{sarwar2001item}}
\end{figure}

De posse da matriz $n \times m$, o algoritmo \textbf{CF} faz a predição/recomendação ao usuário corrente, demonstrado na \autoref{collaborative_based} por $u_{a}$. O usuário $u_{a}$ é visto pelo algoritmo como o alvo atual para o qual serão feitas as predições/recomendações. \citeonline{sarwar2001item} também especificam a predição como um valor numérico que expressa a probabilidade prevista do item ser de interesse do usuário $u_{a}$, sendo este um item ainda não pertencente ao conjunto de $Iu_{a}$. Por outro lado, a recomendação é descrita como uma lista de $\textbf{N}$ itens, cada item $I_{r}$ dentre os itens com a maior probabilidade de utilidade ao usuário ua e ainda não avaliados pelo mesmo. Esta forma de recomendação também é conhecida como recomendação \textbf{Top-N} \cite{adomavicius2005toward}.

Diferentemente do método baseado em conteúdo, a filtragem colaborativa não possui apenas uma abordagem.     Tanto \apudonline{sarwar2001item}{breese1998empirical} quanto \citeonline{adomavicius2005toward} dividem a filtragem colaborativa em duas ramificações:

\begin{itemize}

	\item \textbf{Baseada em memória (\textit{memory-based})}: implica na utilização de toda a matriz $n \times m$ para obter um conjunto de usuários vizinhos (\textit{neighbor-users}), ou seja, usuários que tendem a avaliar diferentes itens similarmente ou itens similares diferentemente ao usuário $u_{a}$. Ao obter o conjunto, os métodos baseados em memória combinam as preferências dos usuários, fornecendo uma recomendação Top-N ao usuário $u_{a}$.

	\item \textbf{Baseada em modelo (\textit{model-based})}: ao invés de utilizar toda a matriz $n \times m$, esta técnica constrói um modelo das avaliações de cada usuário através de diferentes técnicas de \textit{machine learning}, tais como modelos de \textit{cluster} e redes Bayesianas. Devido a complexidade destas técnicas e das mesmas não pertencerem ao escopo da solução apresentada neste trabalho , não abordaremos mais a fundo seu funcionamento.

\end{itemize}

Dessa forma, sistemas de filtragem colaborativa podem ser usados nos casos em que se deseja recomendar itens úteis a um usuário ou fornecer uma previsão ao usuário da probabilidade do mesmo gostar de um item em particular. Além disso, é possível recomendar ao usuário não só itens, mas também usuários ou grupos de usuários que o mesmo possa gostar, o que não é possível nos sistemas baseados em conteúdo \cite{schafer2007collaborative}.

Considerando tais utilidades, tanto \citeonline{schafer2007collaborative} quanto \citeonline{adomavicius2005toward} expõem os sistemas de recomendação baseados em conteúdo e colaborativos como complementares, uma vez que o método baseado em conteúdo prediz a relevância de itens sem avaliações, enquanto o método colaborativo prediz a relevância através de recomendações alheias. A união destas técnicas, em prol de maximizar a eficiência e compensar as limitações (seção 2.2.4), deu origem ao \textbf{método híbrido} que será abordado a seguir.

\subsection{Método Híbrido}

Sistemas de recomendação híbridos seriam quaisquer sistemas que combinam múltiplas técnicas de recomendação para produzir seu resultado \cite{burke2002hybrid, burke2007hybrid}. Como apresentado por \citeonline{adomavicius2005toward}, as técnicas de recomendação possuem limitações de acordo com a abordagem utilizada. Sendo assim, é possível combinar diferentes técnicas para obter o desempenho e precisão desejadas.

\begin{quadro}[h!tp]
	\caption{\label{recommender_systems}Técnicas de recomendação.}

	\begin{center}
		\includegraphics[scale=0.6]{images/recommender_systems.png}
	\end{center}
	\hspace{5.5cm}{Fonte: \citeonline{burke2002hybrid}}
\end{quadro}

Como pode ser visto no quadro 2, \citeonline{burke2002hybrid} apresenta uma série de métodos de recomendação além dos mais comuns abordados neste trabalho. Estes métodos, combinados entre si, podem gerar sistemas híbridos categorizados da seguinte forma:

\begin{itemize}

	\item \textbf{Atribuição de peso (\textit{Weighted}):} Consiste na atribuição de peso para cada um dos métodos empregados no sistema híbrido. Baseado no histórico de acertos entre um método e outro, é possível ajustar o peso de cada um, dando um peso maior ao método atualmente mais eficiente.

	\item \textbf{Escalonamento (\textit{Switching}):} Consiste na utilização de um critério pré-definido para escolher qual método será utilizado no momento. Por exemplo, se o método colaborativo não fornecer uma recomendação com confiança suficiente, o sistema pode trocar para o método baseado em conteúdo.

	\item \textbf{Misto (\textit{Mixed}):} Consiste em usar tanto recomendações de um método quanto de outro, apresentando os resultados de ambos ao usuário.

	\item \textbf{Combinação de características (\textit{Feature Combination}):} Consiste em utilizar a informação colaborativa apenas como características adicionais no conjunto utilizado pelo método baseado em conteúdo.

 	\item \textbf{Cascata (\textit{Cascade}):} Este método em especial consiste em refinamento por estágio, ou seja, o primeiro método é utilizado para gerar um conjunto de recomendações, enquanto o segundo é responsável por refinar o conjunto gerado e assim por diante.

	\item \textbf{Aumento de Recursos (\textit{Feature Augmentation}):} Esta técnica utiliza a recomendação gerada pelo primeiro método como informação para o processamento do segundo método.

	\item \textbf{Meta-nível (\textit{Meta-level}):} Consiste em utilizar o modelo de saída de um método como entrada para o outro. Diferente do aumento de recursos, nesta técnica todo o modelo gerado pelo primeiro método é utilizado.

\end{itemize}

Tendo em vista a taxonomia apresentada por \citeonline{burke2002hybrid}, nota-se que a recomendação híbrida não refere-se ao funcionamento das recomendações, mas sim sobre como os diferentes métodos \textbf{interagem entre si}. Esta interação pode ser insensível à ordem, nos casos de métodos como a atribuição de peso, misto, escalonamento e combinação de características. Já nas outros métodos apresentados, a ordem de execução dos métodos de recomendação alteram o resultado final, uma vez que a saída de um, direta ou indiretamente é a entrada de outro.

Por exemplo, \citeonline{tran2000hybrid} apresentam uma arquitetura híbrida, utilizando os métodos baseado em colaboração (collaborative-based) e baseado em conhecimento (knowledge-based), ambos exemplificados através da arquitetura ilustrada na \autoref{hybrid_scheme}.

\begin{figure}[h!tp]

	\caption{\label{hybrid_scheme}Exemplo de arquitetura híbrida.}

	\begin{center}
		\includegraphics[scale=0.8]{images/hybrid_scheme.png}
	\end{center}

	\hspace{5.5cm}{Fonte: \citeonline{tran2000hybrid}}

\end{figure}

Como ilustrado na \autoref{hybrid_scheme}, a arquitetura descrita exemplifica um sistema híbrido de escalonamento (\textit{switching}). Sendo assim, dependendo da situação atual, o sistema pode trocar entre a recomendação colaborativa e a baseada em conhecimento, visando prover melhores recomendações. Considerando que inicialmente a abordagem colaborativa não seria muito eficiente, enquanto a base de dados não possui muitos usuários com modelos conhecidos e não existem itens avaliados o suficiente, \citeonline{tran2000hybrid} optaram por escalonar para o método baseado em conhecimento.

Através dessas limiares, toda vez que o usuário requisita uma recomendação, o agente de interface interativa (\textit{interactive interface agent}) verifica se as mesmas já foram atendidas. Se sim, o agente utiliza a recomendação do método de filtragem colaborativa, se não, o método baseado em conhecimento é utilizado.

Tanto \citeonline{balabanovic1997fab} quanto \citeonline{claypool1999combining} utilizam sistemas híbridos compostos de duas técnicas combinadas: \textbf{baseado em conteúdo} e \textbf{baseado em colaboração}. Dessa forma, é possível utilizar o método colaborativo para gerar o conjunto de $\textbf{N}$ usuários vizinhos (\textit{neighbor-users}) já descrita neste trabalho. A partir do conjunto gerado é aplicado o método baseado no conteúdo destes usuários próximos, aumentando a precisão da recomendação gerada.

Ao invés de se utilizar apenas um método, a utilização de sistemas híbridos pode trazer uma série de benefícios: ao executar recomendações baseadas em conteúdo, o sistema colaborativo pode lidar com novos usuários que ainda não tem seu modelo definido; torna-se possível fazer recomendações precisas a um usuário, mesmo que não existam usuários similares ao mesmo; pode-se recomendar itens não avaliados por nenhum dos usuários, cruzando o modelo dos mesmos com o conteúdo do item \cite{balabanovic1997fab}.

Como forma de verificar a eficácia do método híbrido em relação aos métodos utilizados de forma individual, \citeonline{claypool1999combining} utilizam como métrica a inexatidão, sendo o termo referente a discrepância entre a recomendação obtida e o resultado esperado. A inexatidão dos métodos em relação a seu tempo de utilização pode ser visto através do resultado ilustrado na \autoref{innacurace}.

\begin{figure}[h!tp]
	\caption{\label{innacurace}Inexatidão entre os métodos de recomendação.}
	\begin{center}
		\includegraphics[scale=0.8]{images/innacurace.png}
	\end{center}
	\hspace{5.5cm}{Fonte: \citeonline{claypool1999combining}}
\end{figure}

Analisando a \autoref{innacurace} pode-se verificar que nos primeiros dias, a inexatidão do método colaborativo era maior devido a falta de completude no modelo dos usuários, construído por meio de usuários que ainda não avaliaram itens, ou de usuários que não se beneficiam da opinião de outros \cite{claypool1999combining}. Conforme o método colaborativo foi estabelecendo relações entre os usuários, este ficou mais preciso e o método baseado em conteúdo começou a ser menos viável. Porém, independente dos picos de inexatidão dos métodos separados mostrados na \autoref{innacurace}, quando combinados (\textbf{recomendação híbrida}), é possível notar uma constância muito maior, possuindo o mais baixo nível de inexatidão em todos os momentos.

Em resumo os sistemas híbridos foram criados para unir técnicas de recomendação com objetivo de \textbf{compensar as limitações} apresentadas pela utilização dessas mesmas técnicas individualmente \cite{balabanovic1997fab}. Tais limitações e seus efeitos no resultado das recomendações serão abordadas na seção a seguir.

\subsection{Limitações}

Conforme apresentado por \citeonline{adomavicius2005toward}, decorrente da utilização das técnicas acima descritas, tanto os sistemas baseados em conteúdo quanto os sistemas colaborativos possuem limitações. Estas limitações, motivo da criação dos sistemas híbridos \cite{balabanovic1997fab}, possuem características claras de acordo com o tipo de recomendação utilizado, sendo divididas da seguinte maneira:

\begin{itemize}

	\item \textbf{Análise de conteúdo limitada (\textit{limited content analysis}):} presente nas técnicas baseadas em conteúdo, devido as mesmas serem limitadas por uma quantidade específica de características relevantes para a recomendação. Além disso, essas características precisam ser extraídas de forma explícita, o que dificulta muito a extração de atributos através de conteúdo como vídeo, imagem, etc.

	\item \textbf{Problema do novo usuario (\textit{new user problem}):} comum nas técnicas que utilizam as preferências do usuário como métrica, consiste no fato de que um usuário precisa ter um número suficiente de avaliações para que o sistema entenda suas preferências e forneça recomendações precisas.

	\item \textbf{Sobre-especialização (\textit{over-specialization}):} comum nas técnicas de recomendação baseada em conteúdo, consiste no fato de que se o sistema apenas recomenda ao usuário itens semelhantes aos que ele já avaliou de forma positiva, o usuário será limitado à apenas recomendações de itens já avaliados, reduzindo cada vez mais a recomendação de novos itens.

	\item \textbf{Problema do novo item (\textit{new item problem}):} sistemas colaborativos baseiam-se apenas nas preferências dos usuários para fazer as recomendações. Sendo assim, novos itens que ainda não foram avaliados por nenhum usuário não serão recomendados.

	\item \textbf{Esparsidade (\textit{Sparsity}):} quando um item é raramente recomendado devido a sua esparsidade no conjunto de usuários e itens, ou seja, um item que é pouco recomendado pelos usuários tende a ser cada vez menos recomendado em sistemas colaborativos, devido ao pouco número de avaliações que o mesmo possui.

\end{itemize}

Sendo assim, grande parte das pesquisas relacionadas a sistemas de recomendação tem como objetivo principal melhorar a precisão das técnicas, reduzindo o impacto das limitações descritas. Porém, como apresentado por \citeonline{mcnee2006being}, nem sempre os itens mais precisos em relação às métricas de cada método são os mais úteis aos usuários.

Considerando que os sistemas de recomendação usualmente abordam apenas algumas, das muitas métricas que definem a utilidade de um item ao usuário, \citeonline{mcnee2006being} ressaltam que cada vez mais a utilização de sistemas de recomendação leva a construção de um conjunto de itens muito similar. Isso ocorre pois quando um usuário avalia um item, as próximas recomendações levarão o mesmo em consideração, recomendando itens cada vez mais parecidos com o item avaliado. Este processo acaba gerando o que \citeonline{mcnee2006being} definem como “\textbf{buraco de similaridade}”, onde o sistema tende a fazer apenas recomendações excepcionalmente similares.

\section{API: \textit{Application Programming Interface}}

Seção sobre API

\subsection{O Padrão RESTful}

Descrever o padrão RESTful

\subsection{Orientação à Metadados}

Encaixar os metadados aqui

%
%--------- FIM REFERENCIAL TEORICO------------
%

\chapter{API DE RECOMENDAÇÃO ORIENTADA À METADADOS}

Amparado pelos benefícios da utilização de mais de um método de recomendação, apresentando-os em forma de serviço, este capítulo detalha o funcionamento de uma \textbf{API como interface de recomendação}, fazendo uso de metadados provenientes do próprio usuário para criação, validação e interconexão entre as recomendações.

Como parte introdutória ao funcionamento, a seção \nameref{visao_geral} traça de forma abstrata as funcionalidades da API, apresentando os principais componentes e sua correlação. Posteriormente, nas seções \nameref{analisador}, \nameref{motor} e \nameref{interface}, tais funcionalidades são dissecadas de forma diminuir o nível de abstração. Por fim, os testes efetuados e as facilidades implementadas para a utilização da solução são abordados na seção \nameref{testes}.

\section{Visão Geral} \label{visao_geral}

A solução construída tem como função recomendar quaisquer itens a quaisquer usuários, sendo estes provenientes de uma fonte externa, fornecidos pelo usuário da API, nos padrões definidos pelos respectivos \textbf{metadados} existentes, também fornecidos inicialmente pelo usuário da API.

O usuário da API é definido como qualquer agente que tenha interações com a aplicação, humano ou programa, através requisições aos \textit{endpoints} fornecidos na documentação. Essas interações são acordadas pelo protocolo HTTP, usado como base em uma arquitetura RESTful \cite{rodriguez2008restful}. Dessa forma, o usuário da API pode utilizar qualquer interface que suporte o protocolo HTTP para desfrutar das funcionalidades da API, tornando-a multiplataforma e independente de bibliotecas de linguagens de programação específicas. As interações com os recursos da aplicação são efetuadas através dos seguintes métodos do protocolo HTTP:

\begin{itemize}

	\item \textbf{GET}: Interações que demandam a consulta de recursos.

	\item \textbf{POST}: Interações que demandam a criação de novos recursos.

	\item \textbf{PUT}: Interações que demandam a alteração de recursos existentes.

	\item \textbf{DELETE}: Interações que demandam a remoção ou desativação de recursos existentes.

\end{itemize}

Após o envio, a requisição é processada pela interface de usuário da API e, posteriormente, um retorno de sucesso ou erro finaliza a requisição. Para verificar o tipo do retorno da requisição, o usuário da API deve se ater aos \textit{status code} retornados pela API. Em caso de sucesso, por exemplo, a API retornaria um conteúdo JSON relacionado a requisição original e um \textit{status code} da faixa \textbf{2XX}. Já em caso de erro, a API retornaria um conteúdo de erro, anexo a um \textit{status code} da faixa \textbf{4XX} \cite{fielding1999hypertext}. Exemplos de requisição à API e o fluxo das mesmas após o envio podem ser visualizados nas figuras \ref{fetch_api}, \ref{postman} e \ref{interconnection}.

\begin{figure}[htp]

	\caption{\label{fetch_api}Exemplo de adição de item via \textit{fetch} API.}

	\begin{center}
		\includegraphics[scale=0.8]{images/fetch_api.png}
	\end{center}

	\hspace{5.5cm}{Fonte: O Autor.}

\end{figure}

\begin{figure}[htp]

	\caption{\label{postman}Exemplo de adição de item via Postman.}

	\begin{center}
		\includegraphics[scale=0.85]{images/postman.png}
	\end{center}

	\hspace{5.5cm}{Fonte: O Autor.}

\end{figure}

\begin{figure}[htp]

	\caption{\label{interconnection}Exemplo de fluxo de uma requisição à API.}

	\begin{center}
		\includegraphics[scale=0.62]{images/interconnection.png}
	\end{center}

	\hspace{5.5cm}{Fonte: O Autor.}

\end{figure}

A figura \ref{interconnection} mostra um exemplo de requisição à API de recomendação. Nela, são requisitados os \textbf{dez} itens de maior similaridade a serem recomendados para o usuário denominado pelo parâmetro variável \textbf{\textit{user\_id}}. Este processo de requisição e resposta pode ser dividido em quatro partes principais:

\begin{enumerate}

	\item \textbf{Envio}: O usuário da API, devidamente autenticado, escolhe o \textit{endpoint} que corresponde as suas necessidades, enviando uma requisição através dos métodos HTTP previamente citados.
	
	\item \textbf{Interpretação}: A API receberá a requisição no devido \textit{endpoint}, interpretando-a e repassando-a através de chamadas internas das devidas funcionalidades.
	
	\item \textbf{Processamento}: Uma vez identificada a ação a ser executada, a API processa os dados (nesse caso as recomendações para o usuário \textbf{\textit{user\_id}}), posteriormente formatando-os em uma estrutura JSON.

	\item \textbf{Retorno personalizado}: Com a estrutura JSON processada em mãos, a API verifica os metadados atuais e retorna, com base nos atributos especificados como visíveis, uma estrutura JSON personalizada pelos desejo do usuário da API.

\end{enumerate}

Partindo da premissa que a solução deve atender modelos genéricos, serão fornecidos na inicialização da API os \textbf{metadados} de usuários, itens e avaliações, correspondendo a estrutura necessária pelo usuário da aplicação. Uma vez que os metadados sejam fornecidos, os dados relacionados devem respeitar as estruturas definidas. Tanto os usuários, itens e avaliações, quanto as futuras recomendações, serão persistidas em um banco de dados, a fim de centralizar as informações e diminuir o tempo de resposta das recomendações requisitadas.

De posse das estruturas de metadados fornecidas na inicialização, o usuário da solução poderá alimentar o sistema através das seguintes interfaces gerais:

\begin{itemize}

	\item \textbf{Metadados}: O usuário da API fornece a estrutura que irá compor cada um dos grupos abaixo. Essas estruturas definem os atributos de cada grupo, o tipo de cada atributo, a importância ou não do atributo nas recomendações, entre outras informações.

	\item \textbf{Usuários}: O usuário da API fornece os usuários aos quais deseja gerar algum tipo de recomendação. Esses dados serão utilizados posteriormente como referência aos itens e para definição da similaridade entre usuários.

	\item \textbf{Itens}: O usuário da API fornece os itens que serão recomendados aos usuários existentes. Os mesmos e seus atributos serão utilizados nas recomendações, atrelados a um usuário em questão.

	\item \textbf{Avaliações}: O usuário da API fornece as avaliações que ligam usuários a itens. Essas informações servem como união, apontando quais usuários estão relacionados a quais itens, sempre atrelando esta relação a uma avaliação numérica.

\end{itemize}

Tais ações de alimentação serão responsáveis por preencher os respectivos conjuntos anteriormente descritos e, a partir deles, construir os modelos de cada usuário e a matriz de avaliações, fundamentais para a geração das recomendações. Durante a requisição de uma recomendação, a solução definirá o método a ser utilizado e o mesmo fará as recomendações com base nos dados conhecidos, respeitando as propriedades descritas nos metadados. Um esquema do funcionamento geral da aplicação pode ser visto na \autoref{overview}.

\begin{figure}[h!tp]

	\caption{\label{overview}Visão geral da API.}

	\begin{center}
		\includegraphics[scale=0.6]{images/MORPY_overview.png}
	\end{center}

	\hspace{5.5cm}{Fonte: O Autor.}

\end{figure}

De modo a complementar a visão da API apresentada pela figura \ref{interconnection}, a figura \ref{overview} apresenta o fluxo interno de funcionamento da API, apresentando uma distinção básica entre dois tipos de eventos: os eventos de \textbf{inserção ou atualização de dados} e os eventos de \textbf{geração e requisição de recomendações}. Ambos estão contidos no domínio de atuação da interface de comunicação com o usuário, a qual interpreta as requisições e processa as devidas respostas. Os módulos centrais correspondentes a estes domínios e suas interações podem ser subdivididas nos seguintes:

\begin{itemize}

	\item \textbf{Interface de Comunicação}: É o ponto de contato entre o cliente que utiliza a API e o domínio da aplicação, recebendo todas as requisições nos devidos \textit{endpoints}, processando-as e retornando os resultados. Este módulo é descrito detalhadamente na seção \nameref{interface}.

	\item \textbf{Analisador de Dados}: É o módulo responsável por contrastar as entradas da interface de comunicação com os metadados ativos. O trabalho do analisador de dados é validar o padrão de dados fornecido pelo cliente da API e persistir os dados corretos na base dados, notificando o motor de recomendação sobre quaisquer mudanças. Este módulo é descrito detalhadamente na seção \nameref{analisador}.

	\item \textbf{Motor de Recomendação}: É o módulo que definitivamente faz as recomendações. Ele se alimenta dos dados persistidos pelo analisador, gerando matrizes de relação, utilizadas para calcular o nível de similaridade entre usuário e itens, permitindo a persistência das recomendações na base de dados. Este módulo é descrito detalhadamente na seção \nameref{motor}.

\end{itemize}

Assim que uma requisição é recebida pela interface de comunicação, se a mesma for uma requisição de recomendação, a interface delega ao \textbf{motor de recomendações}, que busca as recomendações persistidas para o usuário em questãoo. Caso a requisição seja uma alteração de dados existentes, a interface delega ao \textbf{analizador de dados} a tarefa de validar, sempre baseando-se nos metadados atuais, os novos dados providos pelo usuário da API. Assim que esses dados são persistidos na base, o \textbf{motor de recomendações} identifica uma mudança na base e inicia um novo processo paralelo de treinamento para o dado modificado. Dessa forma, quando uma requisição for solicitada, basta o motor de recomendações consultar as recomendações já persistidas.

\section{A Interface de Comunicação} \label{interface}

Visando facilitar a comunicação com o usuário da API através do padrão RESTful, a interface de comunicação tem como principal função ser o ponto de contato entre o agente da requisição e as funcionalidades da API, atendendo a certos padrões de requisição e resposta que serão minuciosamente abordados nesta seção.

Ao passo que a API usa o padrão RESTful como base para a comunicação, é necessário que a mesma utilize um único padrão de dados para que ambos, usuário e API, tenham uma estrutura pré definida de possibilidades de envio e retorno. Dessa forma, o padrão de dados \textbf{JSON}, proposto por \citeonline{crockford2006application} foi o escolhido para servir como representação de dados da API, desde o recebimento de novos recursos (itens, usuários, etc.) até o retorno das recomendações. Um exemplo da representação JSON para as recomendações pode ser visto na figura \ref{json_example}.

\begin{figure}[h!tp]

	\caption{\label{json_example}Exemplo de estrutura JSON para recomendações.}

	\begin{center}
		\includegraphics[scale=0.75]{images/json_example.png}
	\end{center}

	\hspace{5.5cm}{Fonte: O Autor.}

\end{figure}

Em outras palavras, esta seção apresenta o ponto de contato com a API, onde todas as requisições para modificações de recursos são feitas, visando melhorar a precisão das recomendações. Primeiro, é abordado o padrão de nomeclatura dos \textit{endpoints} na seção \ref{endpoints}, de modo a esclarecer de forma geral, toda a gama de possibilidade que o usuário da API tem ao utilizá-la. Em seguida, é apresentado o processo de autenticação na seção \ref{autenticacao}, necessário para que o usuário da API possa executar as requisições. Por fim,... 

\subsection{Padrão de \textit{Endpoints}} \label{endpoints}

Toda a comunicação entre o agente das requisições e a API funciona através de \textit{endpoints} que modificam seus respectivos recursos. Recursos esses que são abstrações da base de dados, fazendo com que o agente das requisições possa interferir diretamente na evolução das recomendações, mesmo sem que haja uma interface gráfica. Um exemplo de recurso pode ser visualizado no quadro \ref{resource_example}.

\begin{quadro}[h!tp]

	\caption{\label{resource_example}Exemplo de recurso de \textit{endpoints} para o módulo item.}
	\centering
	\begin{tabular}{|l|l|l|}
	\hline
	\textbf{Método HTTP} & \textit{\textbf{Endpoint}} & \textbf{Descrição}         \\ \hline
	POST                 & /item                      & Cria um novo item           \\ \hline
	GET                  & /item/:item\_id 			  & Retorna um item específico  \\ \hline
	PUT                  & /item/:item\_id            & Altera um item específico  \\ \hline
	DELETE               & /item/:item\_id            & Deleta um item específico  \\ \hline
	GET                  & /item                      & Retorna todos os itens      \\ \hline
	\end{tabular}

	Fonte: O Autor.

\end{quadro}

Analisando o quadro \ref{resource_example}, nota-se que um mesmo endpoint pode apontar para diferentes funções, uma vez que o método HTTP que encabeça a requisição também é levado em consideração. Além disso, é importante ressaltar que cada par de \textbf{método HTTP} e \textbf{\textit{endpoint}} correspondem diretamente a um \textbf{controlador} da API, que recebe os parâmetros variáveis da requisição (no quadro \ref{resource_example} representando por \textit{item\_id}) e os repassa aos serviços. Sendo assim, os controladores poderiam ser definidos como as "portas de entrada" da API, direcionando a execução de cada funcionalidade. 

\subsection{Autenticação} \label{autenticacao}

Como forma de aumentar a segurança durante as requisições e, principalmente, garantir que apenas os usuários com acesso ao serviço façam requisições, a API conta com um sistema de autenticação via \textit{token}.

Antes da primeira inicialização, o \textit{script} gera um \textit{token} único para o usuário da API, que será posteriormente concatenado com uma palavra secreta nos arquivos de configuração, além da data e hora atuais da requisição de autenticação. O resultado final é um \textit{token} temporário, com validade de até quinze dias, que será utilizado, obrigatoriamente, no cabeçalho de quaisquer requisições aos recursos da API.

Do mesmo modo que o agente das requisições precisa incluir o \textit{token} temporário no cabeçalho das requisições, a API precisa verificar a validade deste \textit{token}. Para que isso seja possível, foi criada uma classe intermediária (\textit{middleware}) que intercepta todas as requisições, exceto a requisição de autenticação. Essa classe, por sua vez, verifica a existência do \textit{token} no cabeçalho da requisição bem como sua validade. Caso o \textit{token} não exista ou não remeta a um usuário de API válido, o \textit{middleware} bloqueia a execução da requisição, retornando uma mensagem de erro e um \textit{status code} igual a \textbf{403} - "\textit{Forbidden}". Um exemplo requisição de autenticação pode ser visto na figura \ref{auth_example}.

\begin{figure}[h!tp]

	\caption{\label{auth_example}Exemplo de autenticação.}

	\begin{center}
		\includegraphics[scale=0.6]{images/auth_example.png}

	\end{center}

	\hspace{5.5cm}{Fonte: O Autor.}

\end{figure}

\section{O Analisador de Dados} \label{analisador}

Para tornar possível a geração de recomendações de forma genérica, através dos metadados, foi necessária a construção de um módulo dedicado exclusivamente a este fator. O \textbf{analisador de dados} tem como principal objetivo cruzar as novas informações, repassadas pela interface de comunicação, com os metadados correntes.

Por exemplo, para que um novo item seja adicionado ao conjunto de itens conhecidos e, posteriormente, treinado contra os outros itens, é necessário que o analisador de dados verifique se todos os atributos relevantes para as recomendações estão preenchidos, se os tipos dos campos fornecidos condizem com os tipos especificados nos metadados, etc. 

Em síntese, esta seção apresentará os principais componentes do analisador de dados. Primeiro apresentando os padrões utilizados para a orientação a metadados na seção \ref{analisador:padrao_metadados}, em seguida, os modelos gerados pelos metadados e sua utilização em toda a aplicação, aprofundados na seção \ref{analisador:modelos} e, por fim, como é feita a persistência destes dados na seção \ref{analisador:persistencia}.

\subsection{Padrão de Metadados} \label{analisador:padrao_metadados}

Como parte central do diferencial de funcionamento da aplicação esta a orientação a metadados. Como abordado na seção \ref{}, os metadados são eficientes na construção de estruturas de dados construídas dinamicamente, incertas até sua execução. Sendo assim, os mesmos formam a base que sustenta os dados desta aplicação.

Uma vez que não se sabe qual o padrão de JSON a se retornar ao usuário, ou quais os campos relevantes para a geração das recomendações, ou ainda quais sequer são os nomes dos atributos, os metadados são as estruturas, previamente definidas e passíveis de personalização, fornecidas pelo usuário na inicialização da aplicação. Estas estruturas são enviadas através de uma estrutura JSON para os \textit{endpoints} respectivos, divididos em três categorias:

\begin{itemize}
	
	\item \textbf{Usuários}: Definem os nomes dos atributos, obrigatórios ou não, que compõem o modelo de usuário, os tipos de cada atributo, a quantidade máxima de caracteres, se é recomendável ou não e seu peso. Além disso os metadados de usuários apontam o identificador único do usuário.

	\item \textbf{Itens}: Assim como os usuários, definem os nomes de cada atributo, obrigatórios ou não, do item e suas demais características. Além disso, é definida a exibição de cada atributo do item no retorno das recomendações e seu identificador único.

	\item \textbf{Avaliações}: Definem obrigatoriamente o campo de identificador único do usuário e o campo de identificador único do item, amarrando efetivamente as duas entidades. Além disso, definem o tipo de avaliação que será empregada (binária ou não).

\end{itemize}

Para ilustrar o processo de definição dos metadados, a figura \ref{metadata} mostra um exemplo real dos metadados inicais para o modelo de \textbf{item}. Estes atributos podem ser modificados posteriormente através do mesmo \textit{endpoint} utilizando o método PUT do HTTP. Caso o qualquer atributo sofra posterior alteração, os metadados são salvos com o status \textbf{"\textit{active:false}"}, orientado que o mesmo não é mais válido. Em seguida, um novo conjunto de metadados é criado com o atributo \textbf{"\textit{active:true}"} e as modificações solicitadas pelo usuário.

\begin{figure}[htp]

	\caption{\label{metadata}Exemplo de declaração de metadados para o item.}

	\begin{center}
		\includegraphics[scale=0.8]{images/metadata.png}
	\end{center}

	\hspace{5.5cm}{Fonte: O Autor.}

\end{figure}

Analisando a figura \ref{metadata} nota-se um conjunto de chaves pré-definidas para a estrutura declarada. A razão dessas chaves possuirem suas definições \textbf{obrigatórias e pré-definidas} é o fato de que as mesmas moldam todas as outros atributos dinamicamente atribuidos aos modelos de itens, usuários e avaliações, consequentemente moldando o banco de dados. Sendo assim, cada chave da estrutura definida na figura é utilizada como atributo chave para alguma decisão estrutural dentro da API. As chaves definidas na figura possuem as seguintes funcionalidades:

\begin{itemize}
	\item \textbf{\textit{type}}: Fora da chave \textbf{\textit{"attributes"}}, define o tipo do grupo de metadados que correspondem os atributos declarados. Dentro de um atributo, define o tipo do dado ao qual o atributo espera um valor, podendo ser qualquer tipo natual: \textbf{\textit{string}}, \textbf{\textit{integer}}, \textbf{\textit{float}}, etc. Espera um valor do tipo \textbf{\textit{string}}.
	
	\item \textbf{\textit{attributes}}: Declara a lista de atributos percentes ao \textit{type}, neste caso ao item.  Espera um valor do tipo \textbf{\textit{array}}.
	
	\item \textbf{\textit{name}}: Define o nome de um atributo, o qual será utilizado como chave do atributo no banco de dados e na construção dinamica do modelo. Espera um valor do tipo \textbf{\textit{string}}.

	\item \textbf{\textit{key}}: Define se o atributo é ou não um identificador único. Será utilizado posteriormente para ligar os usuários a itens e para fazer as consultas dos endpoints.  Espera um valor do tipo \textbf{\textit{boolean}}.
	
	\item \textbf{\textit{hide}}: Define se o atributo será ou não exibido no JSON de retorno. Espera um valor do tipo \textbf{\textit{boolean}}.

	\item \textbf{\textit{unique}}: Define se o atributo deve ser único na base de dados ou não. Caso o seu valor seja verdadeiro (\textit{true}), o analisador de dados retornará um erro para a requisição, caso este atributo já exista. Espera um valor do tipo \textbf{\textit{boolean}}.

	\item \textbf{\textit{nullable}}: Por padrão, todos os atributos declarados nos metadados serão obrigatórios na inserção de novos registros. Caso este atributo estiver com o valor verdadeiro (\textit{true}), o mesmo torna-se opcional. Espera um valor do tipo \textbf{\textit{boolean}}.

	\item \textbf{\textit{max\_length}}: Declara a quantidade máxima de tamanho/caracteres que o campo suporta. Será utilizada posteriormente para barrar entradas maiores que esse valor no banco de dados. Espera um valor do tipo \textbf{\textit{integer}}.

	\item \textbf{\textit{recommendable}}: Define se o atributo será utilizado como base para as recomendações ou não. Será utilizado pelo motor de recomendações para levar em consideração apenas os atributos que o usuário da API julgar relevantes. Espera um valor do tipo \textbf{\textit{boolean}}.

	\item \textbf{\textit{weight}}: Se o attributo \textbf{\textit{recommendable}} for verdadeiro (\textit{true}), define o a peso do atributo nos cálculos de recomendação. O peso varia de um a dez. Espera um valor do tipo \textbf{\textit{integer}}.
\end{itemize}

Assim que a API toma conhecimento destes metadados, já é possível ao usuário da API alimentar a base de dados com seus itens iniciais.

\subsection{Modelos de Itens, Usuários e Avaliações} \label{analisador:modelos}

Explicar que eles são dinamicos e com base nos metadados. Explicar os métodos que a classe tem. Apontar pro cara ver nos apendices o código completo.

\subsection{Persistência} \label{analisador:persistencia}

A base de um sistema de recomendação é o conhecimento prévio, uma vez que, independente da métrica ou método utilizados, todos utilizam do cruzamento de informações conhecidas para gerar predições (vide seção \ref{recommender_systems}). Dessa forma, é necessário que alguma estratégia eficiente de armazenamento e consulta seja utilizada, tendo em vista que o cruzamento dessas informações pode gerar matrizes excedendo milhões linhas e colunas \cite{gomez2016netflix}.

Visando utilizar uma estratégia de persistência que se destacasse no processamento de grandes conjuntos de dados, ponto essencial para boas recomendações, a API implementa uma persistência utilizando um banco de dados não relacional (NoSQL) orientado a documentos \cite{leavitt2010will}, que além de sua proeficientia no manejo dos dados, também persiste os mesmos em estruturas baseadas em JSON, como apresentado por \citeonline{padhy2011rdbms}.

Criar outra figura pra explicar as 3 partes: requisição do usuário via HTTP com os dados pra atualizar, persistencia e retorno | em paralelo: geração e persistência das recomendações e retorno das recomendações em outra requisição futura.

Explicar a escolha do paradigma não relacional devido a quantidade de dados e o padrão ser parecido com JSON. Referenciar BSON

\section{O Motor de Recomendações} \label{motor}

Como forma de recomendação será utilizado o método híbrido, composto dos métodos \textbf{baseado em conteúdo} e \textbf{filtragem colaborativa}, escalonando através do método com melhor precisão momentânea. Dessa forma é possível atender qualquer tipo de metadado, fornecendo recomendações independente do número de usuários, itens e avaliações na base de dados.

Em um primeiro momento, a API utilizará o método baseado em conteúdo para fornecer as recomendações e predições, uma vez que poucos itens estarão avaliados e o método colaborativo não terá modelos de usuários suficientes. Assim que as métricas de relações entre usuários e quantidade de modelos processados sejam supridas, a API passará a utilizar o método de filtragem colaborativa, caso este esteja gerando recomendações mais precisas.

\subsection{Motor de Conteúdo} \label{motor:conteudo}

\subsection{Motor Colaborativo} \label{motor:colaborativo}

\section{Testes e Implementação} \label{testes}

\subsection{\textit{Seed} de Dados}

\subsection{Tecnologias Utilizadas} \label{tecnologias}

%
%--------- FIM API------------
%

\chapter{RESULTADOS E DISCUSSÃO}

\section{Trabalhos Futuros} \label{trabalhos_futuros}

\chapter{CONSIDERAÇÕES FINAIS}

Devido ao grande número de implementações dos sistemas de recomendação nas mais diversas áreas que aqui foram apresentadas, torna-se notável a vasta gama de aplicações das técnicas e, mais do que isso, a necessidade de um serviço multi-propósito desprendido do uso de linguagens de programação específicas. Sendo assim, ao final do cronograma, espera-se a obtenção de uma API que satisfaça estes requisitos.

Além disso, é importante ressaltar preocupação na construção de uma documentação direta e coesa ao longo de todo o processo, visando uma futura colaboração externa da comunidade, sendo a API de código aberto. Desta forma, o trabalho pode servir não só como uma alternativa \textit{open-source} aos sistemas de recomendação, mas também como uma tecnologia de utilização simples para futuros estudos na área.

\postextual
\bibliography{referencias}
\end{document}